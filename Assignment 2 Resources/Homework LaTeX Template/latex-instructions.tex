\documentclass{article}
% Package and macro definitions for CS 5963/6360: Virtual Reality.
% Originally prepared August 23, 2012 by Jon Doyle

%%% Page dimensions
\setlength{\oddsidemargin}{0in}
\setlength{\evensidemargin}{0in}
\setlength{\topmargin}{0in}
\setlength{\textheight}{9in}
\setlength{\textwidth}{6.5in}
\setlength{\headheight}{0in}
\setlength{\headsep}{0in}
\setlength{\footskip}{0.5in}

%%% Font and symbol definition packages
\usepackage{helvet} 
\usepackage{alltt}
\usepackage{amsfonts, amsmath}
\usepackage{amssymb}

%%% The modified Sellinger fitch.sty file

\newcommand{\Z}{\mathbb{Z}}
\newcommand{\Q}{\mathbb{Q}}
\newcommand{\R}{\mathbb{R}}
\newcommand{\N}{\mathbb{N}}
\def\land{\wedge}
\def\lor{\vee}
\def\implies{\rightarrow}
\def\iff{\leftrightarrow}
\def\turn{\vdash}
\def\Cn{\text{Cn}}
\def\Th{\text{Th}}
\def\defeq{\stackrel{\rm def}{=}}

%%% The environment for providing answers to problems
\newenvironment{answer}%
{\par\noindent\textbf{Answer}\par\noindent}%
{}


%================================================================
% Document Header

\title{CS 5963/6360 \LaTeX~Macros}
\author{Rogelio E. Cardona-Rivera}
\date{January 29, 2019}

%================================================================
% Document Body

\begin{document}
\maketitle

In this class, when you are asked to typeset homework you may do
so in \LaTeX.  Some of the macros come from standard packages, 
and some from local definitions.  As the sources of this file 
show, all the supplied packages and macro definitions can be 
loaded by means of the \LaTeX command 
\verb+% Package and macro definitions for CS 5963/6360: Virtual Reality.
% Originally prepared August 23, 2012 by Jon Doyle

%%% Page dimensions
\setlength{\oddsidemargin}{0in}
\setlength{\evensidemargin}{0in}
\setlength{\topmargin}{0in}
\setlength{\textheight}{9in}
\setlength{\textwidth}{6.5in}
\setlength{\headheight}{0in}
\setlength{\headsep}{0in}
\setlength{\footskip}{0.5in}

%%% Font and symbol definition packages
\usepackage{helvet} 
\usepackage{alltt}
\usepackage{amsfonts, amsmath}
\usepackage{amssymb}

%%% The modified Sellinger fitch.sty file

\newcommand{\Z}{\mathbb{Z}}
\newcommand{\Q}{\mathbb{Q}}
\newcommand{\R}{\mathbb{R}}
\newcommand{\N}{\mathbb{N}}
\def\land{\wedge}
\def\lor{\vee}
\def\implies{\rightarrow}
\def\iff{\leftrightarrow}
\def\turn{\vdash}
\def\Cn{\text{Cn}}
\def\Th{\text{Th}}
\def\defeq{\stackrel{\rm def}{=}}

%%% The environment for providing answers to problems
\newenvironment{answer}%
{\par\noindent\textbf{Answer}\par\noindent}%
{}
+. That file then loads others.

You are \textbf{\textit{strongly}} encouraged to typeset your answers for 
readability, but you need not typeset your answers in \LaTeX. 
All answer submissions should include a PDF version of the 
answers. You are encouraged to prepare your answers by simply 
adding them into a copy of the homework source.  If you decide
to do so, be sure to replace the name in the author environment:

\begin{verbatim}
\author{Rogelio E. Cardona-Rivera}
\end{verbatim}

\noindent Further, use the constructs \verb+\begin{answer}+ and 
\verb+\end{answer}+ to begin and end your answer.  For example, 
the code,

\begin{verbatim}
\begin{answer}
I think, therefore I am.
\end{answer}
\end{verbatim}

\noindent produces the result:\\[5pt]

\begin{answer}
I think, therefore I am.
\end{answer}


\section{Macros for symbols}

\begin{center}
  \begin{tabular}[h]{lll}
    $\N$ & \verb+\N+ & Natural numbers \\
    $\R$ & \verb+\R+ & Real numbers \\
    $\Z$ & \verb+\Z+ & Integers \\
  \end{tabular}
\end{center}

\end{document}

%%% Local Variables: 
%%% mode: latex
%%% TeX-master: t
%%% End: 
